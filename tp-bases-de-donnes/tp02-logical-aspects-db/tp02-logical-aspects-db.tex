\documentclass{../../cs-classes/cs-classes}

\title{TD 02 - Logical aspect of databases}
\author{Antoine Groudiev}

\newcommand*{\cinema}{\textnormal{Cinema}}
\newcommand*{\titl}{\textnormal{Title}}
\newcommand*{\movie}{\textnormal{Movie}}
\newcommand*{\seen}{\textnormal{Seen}}
\newcommand*{\likes}{\textnormal{Likes}}
\newcommand*{\name}{\textnormal{Name}}
\newcommand*{\tim}{\textnormal{Time}}
\newcommand*{\actor}{\textnormal{Actor}}
\newcommand*{\director}{\textnormal{Director}}
\newcommand*{\produced}{\textnormal{Produced}}
\newcommand*{\producer}{\textnormal{Producer}}
\newcommand*{\viewer}{\textnormal{Viewer}}

\newcommand{\constant}[1]{\textnormal{\say{#1}}}
\usepackage{relsize}
\usepackage{CJKutf8}

\begin{document}
\begin{exercise}
    We suggest the following changes:
    \begin{itemize}
        \item Add a unique identifier to \texttt{Movie}
        \item Replace the \texttt{Title} field in \texttt{Cinema}, \texttt{Produced}, \texttt{Seen} and \texttt{Likes} by a \texttt{Movie} identifier
    \end{itemize}
\end{exercise}

% PSJR = Project, Select, Join & Rename
\begin{exercise}
    Consider the following queries:
    \begin{enumerate}
        % Where and when can one see the movie “Mad Max”?
        \item PSJR algebra:
        \begin{equation*}
            \Pi_{\name, \tim}\left(\sigma_{\titl=\constant{Mad Max}}(\cinema)\right)
        \end{equation*}
        Conjunctive calculus:
        \begin{equation*}
            \cinema(x_{\titl}, x_{\name}, \constant{Mad Max})
        \end{equation*}

        % What are the titles of the movies directed by Orson Welles?
        \item PSJR algebra:
        \begin{equation*}
            \Pi_{\titl}\left(\sigma_{\director =\constant{Orson Welles}}(\movie)\right)
        \end{equation*}
        Conjunctive calculus:
        \begin{equation*}
            \exists x, \movie(x_{\titl}, \constant{Orson Welles}, x)
        \end{equation*}

        % Who are the actors playing in “Ran”?
        \item PSJR algebra:
        \begin{equation*}
            \Pi_{\actor}\left(\sigma_{\titl=\constant{Ran}}\right)
        \end{equation*}
        Conjunctive calculus:
        \begin{equation*}
            \exists x, \movie(\constant{Ran}, x, x_{\actor})
        \end{equation*}

        % Where can one see a movie in which Signoret plays?
        \item PSJR algebra:
        \begin{equation*}
            \Pi_{\name}\left(\Pi_{\titl}(\sigma_{\actor=\constant{Signoret}}(\movie)\underset{\titl=\cinema, \titl}{\bowtie}\cinema)\right)
        \end{equation*}
        Conjunctive calculus:
        \begin{equation*}
            \exists x_{\titl} \;(\exists x \;\movie(x_{\titl}, x, \constant{Signoret})) \land (\exists x \; \cinema(x_{\name}, x, x_{\titl}))
        \end{equation*}

        % Who are the actors that produced a movie?
        \item PSJR algebra:
        \begin{equation*}
            \Pi_{\actor}\left(\movie\underset{\movie.\actor=\produced.\produced}{\mathlarger\bowtie}\produced\right)
        \end{equation*}
        Conjunctive calculus:
        \begin{equation*}
            (\exists x, \exists y, \movie(x, y, x_{\actor})) \land (\exists x, \produced(x_{\actor}, x))
        \end{equation*}

        % Who are the actors that produced a movie in which they play?
        \item PSJR algebra:
        \begin{equation*}
            \Pi_{\actor}\left(\movie\underset{\substack{\movie.\actor=\produced.\producer\\\movie.\titl=\produced.\titl}}{\mathlarger\bowtie}\produced\right)
        \end{equation*}
        Conjunctive calculus:
        \begin{equation*}
            \exists x_{\titl} \; (\exists d \; \movie(x_{\titl}, d, x_{\actor}) \land \produced(x_{\actor}, x_{\titl}))
        \end{equation*}

        % Which actors play in a movie Orson Welles plays in?
        \item PSJR algebra:
        \begin{equation*}
           \Pi_{\actor}\left(\Pi_{\titl}(\sigma_{\actor=\constant{Orson Welles}}(\movie))\underset{\titl=\movie.\titl}{\bowtie}\movie\right)
        \end{equation*}
        Conjunctive calculus:
        \begin{equation*}
            \exists x_{\titl}\; \big((\exists d \; \movie(x_{\titl}, d, \constant{Orson Welles})) \land (\exists d \; \movie(x_{\titl}, d, x_{\actor}))\big)
        \end{equation*}

        % Which producers produce all the movies directed by Akira Kurosawa ?
        \item This query is impossible in PSJR algebra since it is not monotone. Indeed, consider $D_0$ a database on the given schema, containing a movie 
        \begin{equation*}
            \movie(\constant{Citizen Kane}, \constant{Orson Welles}, \constant{Orson Welles})
        \end{equation*}
        and a producer
        \begin{equation*}
            \produced(\constant{Orson Welles}, \constant{Citizen Kane})
        \end{equation*}
        In this database, the query \say{Which producers produce all the movies directed by Akira Kurosawa?} returns the set $\{\constant{Orson Welles}\}$. Now, consider $D_1$ the database $D_0$ to which we added the following movie:
        \begin{CJK}{UTF8}{min}
        \begin{equation*}
            \movie(\constant{七人の侍}, \constant{Akira Kurosawa}, \constant{Toshiro Mifune})
        \end{equation*}
        \end{CJK}
        In this database, the query \say{Which producers produce all the movies directed by Akira Kurosawa?} returns the empty set. Therefore, this query is not monotone, and cannot be expressed in PSJR algebra and conjunctive calculus.
    \end{enumerate}
\end{exercise}

\begin{exercise}
    Consider the following queries:
  \begin{enumerate}
    % Which viewers watch all the movies?
    \item PSJRU algebra:
    \begin{equation*}
        \Pi_{\viewer}(\seen)\setminus\Pi_{\viewer}(\movie\setminus(\movie\bowtie\seen))
    \end{equation*}
    Conjunctive calculus:
    \begin{equation*}
        \forall m, \; \seen(x_{\viewer}, m)
    \end{equation*}
    
    % Which viewers like all the movies they watch?
    \item PSJRU algebra:
    \begin{equation*}
        \Pi_{\viewer}(\seen)\setminus\Pi_{\viewer}(\seen\setminus\likes)
    \end{equation*}
    Conjunctive calculus:
    \begin{equation*}
        \forall m \; (\seen(x_{\viewer}, m)\land \likes(x_{\viewer}, m)) \lor (\lnot \seen(x_{\viewer}, m) \land \lnot \likes(x_{\viewer}, m))
    \end{equation*}
    
    % Who produces a movie that does not play in any cinema?
    \item PSJRU algebra:
    \begin{equation*}
        \Pi_{\producer}(\produced)\setminus\Pi_{\producer}(\movie\bowtie\produced\bowtie\cinema)
    \end{equation*}
    Conjunctive calculus:
    \begin{equation*}
        \exists m \; \underbrace{\produced(x_{\producer}, m)}_{x_{\producer} \textnormal{ produces a movie } m} \land \underbrace{\lnot(\exists n, \; \exists t, \; \cinema(n, t, m))}_{\textnormal{and this movie does not play in theater}}
    \end{equation*}
    
    % Which producers see all the movies they produce?
    \item PSJRU algebra:
    \begin{equation*}
        \Pi_{\producer}(\seen)\setminus\Pi_{\produced}(\seen\setminus\produced)
    \end{equation*}
    Conjunctive calculus:
    \begin{equation*}
        \forall m, \; \lnot(\produced(x_{\producer}, m)) \lor \seen(x_{\producer}, m)
    \end{equation*}
    
    % Which actors have a Bacon number?
    \item Impossible in RSJRU algebra.

  \end{enumerate}  
\end{exercise}

\begin{exercise}
    Division can be expressed using the other operators of PSJRUC algebra:
    \begin{equation*}
        \Pi_1(I)\setminus \underbrace{\Pi_1(\overbrace{\Pi_1(I)\times J}^{\mathclap{\text{every possible $(i, j)$ couple}}} \setminus I)}_{\mathclap{\text{every $x$ such that $\exists y$ with $(x, y)\notin I$}}}
    \end{equation*}
\end{exercise}

\setcounter{exercise}{6}
\begin{exercise}
    \begin{enumerate}
        \item An equivalent relational calculus expression is:
        \begin{equation*}
            \Pi_{R.A}(R)\setminus\Pi_{R.A}(\sigma_{S.C\lor R.B\leq 1}(R\times S))
        \end{equation*}

        An equivalent relational algebra expression is:
        \begin{equation*}
            \exists x_B, \; R(r_A, r_b) \land r_b > 1 \land \lnot (\exists s_c, \; \exists s_B, S(s_C, s_B)\land s_C=r_A)
        \end{equation*}

        \item An equivalent relational calculus expression is:
        \begin{equation*}
            \Pi_A(R)\setminus \Pi_{A_1}(\sigma_{A_1<A_2}(\rho_{A\to A_1}(R)\times \rho_{A\to A_2}(R)))
        \end{equation*}

        An equivalent relational algebra expression is:
        \begin{equation*}
            \underbrace{\big(\exists b, \; R(x_{\max}, b)\big)}_{x_{\max} \text{ is in } A} \land \underbrace{\big(\forall a, \; (\exists b, \; R(a, b)) \land x_{\max} \geq a\big)}_{\text{and } x_{\max} \text{ is the maximum of } A}
        \end{equation*}
    \end{enumerate}
\end{exercise}

\end{document}