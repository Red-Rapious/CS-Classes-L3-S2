\documentclass{../../cs-classes/cs-classes}

\title{TD 02 - Logical aspect of databases}
\author{Antoine Groudiev}

\newcommand*{\cinema}{\textnormal{Cinema}}
\newcommand*{\titl}{\textnormal{Title}}
\newcommand*{\movie}{\textnormal{Movie}}
\newcommand*{\seen}{\textnormal{Seen}}
\newcommand*{\likes}{\textnormal{Likes}}
\newcommand*{\name}{\textnormal{Name}}
\newcommand*{\tim}{\textnormal{Time}}
\newcommand*{\actor}{\textnormal{Actor}}
\newcommand*{\director}{\textnormal{Director}}
\newcommand*{\producer}{\textnormal{Producer}}
\newcommand{\constant}[1]{\textnormal{\say{#1}}}
\usepackage{relsize}
\usepackage{CJKutf8}

\begin{document}
\begin{exercise}
    We suggest the following changes:
    \begin{itemize}
        \item Add a unique identifier to \texttt{Movie}
        \item Replace the \texttt{Title} field in \texttt{Cinema}, \texttt{Produced}, \texttt{Seen} and \texttt{Likes} by a \texttt{Movie} identifier
    \end{itemize}
\end{exercise}

% PSJR = Project, Select, Join & Rename
\begin{exercise}
    Consider the following queries:
    \begin{enumerate}
        % Where and when can one see the movie “Mad Max”?
        \item PSJR algebra:
        \begin{equation*}
            \Pi_{\name, \tim}\left(\sigma_{\titl=\constant{Mad Max}}(\cinema)\right)
        \end{equation*}
        Conjunctive calculus:
        \begin{equation*}
            \cinema(x_{\titl}, x_{\name}, \constant{Mad Max})
        \end{equation*}

        % What are the titles of the movies directed by Orson Welles?
        \item PSJR algebra:
        \begin{equation*}
            \Pi_{\titl}\left(\sigma_{\director =\constant{Orson Welles}}(\movie)\right)
        \end{equation*}
        Conjunctive calculus:
        \begin{equation*}
            \exists x, \movie(x_{\titl}, \constant{Orson Welles}, x)
        \end{equation*}

        % Who are the actors playing in “Ran”?
        \item PSJR algebra:
        \begin{equation*}
            \Pi_{\actor}\left(\sigma_{\titl=\constant{Ran}}\right)
        \end{equation*}
        Conjunctive calculus:
        \begin{equation*}
            \exists x, \movie(\constant{Ran}, x, x_{\actor})
        \end{equation*}

        % Where can one see a movie in which Signoret plays?
        \item PSJR algebra:
        \begin{equation*}
            \Pi_{\name}\left(\Pi_{\titl}(\sigma_{\actor=\constant{Signoret}}(\movie)\underset{\titl=\cinema, \titl}{\bowtie}\cinema)\right)
        \end{equation*}
        Conjunctive calculus:
        \begin{equation*}
            \exists x_{\titl} \;(\exists x \;\movie(x_{\titl}, x, \constant{Signoret})) \land (\exists x \; \cinema(x_{\name}, x, x_{\titl}))
        \end{equation*}

        % Who are the actors that produced a movie?
        \item PSJR algebra:
        \begin{equation*}
            \Pi_{\actor}\left(\movie\underset{\movie.\actor=\producer.\producer}{\mathlarger\bowtie}\producer\right)
        \end{equation*}
        Conjunctive calculus:
        \begin{equation*}
            (\exists x, \exists y, \movie(x, y, x_{\actor})) \land (\exists x, \producer(x_{\actor}, x))
        \end{equation*}

        % Who are the actors that produced a movie in which they play?
        \item PSJR algebra:
        \begin{equation*}
            \Pi_{\actor}\left(\movie\underset{\substack{\movie.\actor=\producer.\producer\\\movie.\titl=\producer.\titl}}{\mathlarger\bowtie}\producer\right)
        \end{equation*}
        Conjunctive calculus:
        \begin{equation*}
            \exists x_{\titl} (\exists x \; \movie(x_{\titl}, x, x_{\actor}) \land \producer(x_{\actor}, x_{\titl}))
        \end{equation*}

        % Which actors play in a movie Orson Welles plays in?
        \item PSJR algebra:
        \begin{equation*}
           \Pi_{\actor}\left(\Pi_{\titl}(\sigma_{\actor=\constant{Orson Welles}}(\movie))\underset{\titl=\movie.\titl}{\bowtie}\movie\right)
        \end{equation*}
        Conjunctive calculus:
        \begin{equation*}
            \exists x_{\titl}\; \left(\exists x\; \movie(x_{\titl}, x, \constant{Orson Welles}) \land (\exists x \; \movie(x_{\titl}, x, x_{\actor}))\right)
        \end{equation*}

        % Which producers produce all the movies directed by Akira Kurosawa ?
        \item This query is undecidable since it is not monotone. Indeed, consider $D_0$ a database on the given schema, containing a movie 
        \begin{equation*}
            \movie(\constant{Citizen Kane}, \constant{Orson Welles}, \constant{Orson Welles})
        \end{equation*}
        and a producer
        \begin{equation*}
            \producer(\constant{Orson Welles}, \constant{Citizen Kane})
        \end{equation*}
        In this database, the query \say{Which producers produce all the movies directed by Akira Kurosawa?} returns the set $\{\constant{Orson Welles}\}$. Now, consider $D_1$ the database $D_0$ to which we added the following movie:
        \begin{CJK}{UTF8}{min}
        \begin{equation*}
            \movie(\constant{七人の侍}, \constant{Akira Kurosawa}, \constant{Toshiro Mifune})
        \end{equation*}
        \end{CJK}
        In this database, the query \say{Which producers produce all the movies directed by Akira Kurosawa?} returns the empty set. Therefore, this query is not monotone, and cannot be expressed in PSJR algebra and conjunctive calculus.
    \end{enumerate}
\end{exercise}

\begin{exercise}
    Consider the following queries:
  \begin{enumerate}
    % Which viewers watch all the movies?
    \item PSJR algebra:
    \begin{equation}
        t
    \end{equation}
    Conjunctive calculus:
    \begin{equation}
        t
    \end{equation}
    
    %Which viewers like all the movies they watch?
    \item PSJR algebra:
    \begin{equation}
        t
    \end{equation}
    Conjunctive calculus:
    \begin{equation}
        t
    \end{equation}
    
    %Who produces a movie that does not play in any cinema?
    \item PSJR algebra:
    \begin{equation}
        t
    \end{equation}
    Conjunctive calculus:
    \begin{equation}
        t
    \end{equation}
    
    %Which producers see all the movies they produce?
    \item PSJR algebra:
    \begin{equation}
        t
    \end{equation}
    Conjunctive calculus:
    \begin{equation}
        t
    \end{equation}
    
    %Which actors have a Bacon number?
    \item PSJR algebra:
    \begin{equation}
        t
    \end{equation}
    Conjunctive calculus:
    \begin{equation}
        t
    \end{equation}

  \end{enumerate}  
\end{exercise}

\end{document}